\chapter{Introducción Específica} % Main chapter title
%----------------------------------------------------------------------------------------
%	SECTION 1
%----------------------------------------------------------------------------------------
A fin de comprender las decisiones que se adoptaron al momento de definir la arquitectura del software, es necesario comprender el entorno del problema, los casos de uso y requerimientos del sistema. 

\section{Problemática}
\subsection{Entorno del sistema}

En la linea de galvanizado original los técnicos operaban en un ambiente altamente corrosivo por los vapores emanados de las soluciones, y con en el soporte de instrumental de medición muy básico para utilizar en algunos de las etapas en la linea.
En las diferentes bateas de la linea de galvanizado se necesitaban controlar distintos combinación de los siguientes parámetros:
\begin{itemize}
	\item Temperatura.
	\item Nivel de líquidos.
	\item Conductividad o concentración de iones.
	\item Corriente entregada (electrolisis).
	\item Inyección de aire.
\end{itemize}

Debido a que la linea cuenta con varias bateas distintas colocadas en forma consecutiva se planteo la necesidad de en primer lugar de que el sistema debía ser capaz de manejar, en función de parámetros específicos a cada etapa, el conjunto de estos parámetros como un solo núcleo de procesamiento con los sensores y actuadores colocados en una sola cuba de prueba. A su vez era necesario contar con la capacidad de manejar diversas señales de interruptores y actuadores de interacción con el operario.
En función del éxito de este banco la expansión a futuro con módulos de entrada y salida a lo largo de la planta, se implementaría en otra etapa.

% Agregar imagen diagrama con un el sistema prototipo.
\subsection{Arquitectura}



\subsection{Casos de Uso}

Como primera instancia fue necesario conocer los casos de uso a los cuales vincularan al usuario y al sistema en funcion de la operacion a ejecutar. Para ello se definieron tres casos globales:

\begin{itemize}
	\item Puesta en alta del dispositivo.
	\item Operación en modo de funcionamiento 1.
	\item Operación en modo de funcionamiento 2.
\end{itemize}



\begin{table}[h]
\centering
\begin{tabular}{|m{10em}|m{25em}|}
\hline
\multicolumn{1}{|c|}{\textbf{Título}} & \multicolumn{1}{c|}{\textbf{Descripción}} \\ \hline
	1. Nombre & Puesta en alta del dispositivo. \\ \hline
	1.1 Breve descripción  & Secuencia de instalación de dispositivo en las cubas de control. \\ \hline
	1.2 Actor principal  & Técnico instalador. \\ \hline
	1.3 Disparadores  & Alimentación de la unidad. \\ \hline
	2. Flujo de eventos &  \\ \hline
	2.1 Flujo básico & 2.1.1  El sistema inicia el ciclo de encendido y se prepara para evaluar los rangos de las entradas de una por vez. 
	2.1.2  Para cada uno emite una señal a definir en caso de ser correcto y otra en caso de ser incorrecto. 
	2.1.3  Censa que los valores de corriente sobre los lazos de control de conductividad sean los correctos, según el máximo y el mínimo admisible.(a definir) 
	2.1.4  Censa que los valores de los termostatos sean los correctos. (a definir)
	2.1.5  Censa los valores de las entrada de nivel, que deben estar dentro de los establecidos según RINB242? \\ \hline
	2.2 Flujo alternativo  & 2.1.6 Si alguno de los sensores no esta dentro de los valores se bloquea el proceso hasta que se seleccione ignorar o se normalice el parámetro.  \\ \hline
	2.3 Flujo alternativo  &  \\ \hline
	3. Requerimientos especiales & Botón de encendido liberado. Selector de modo de funcionamiento en posición. \\ \hline
	4. Pre-Condiciones  & Interfaces de sensores conectados.  \\ \hline
	5. Pos-Condiciones & El dispositivo queda en modo espera a la señal de inicio de ciclo, midiendo según los requerimientos RFTEM112,121,131,153? y mostrándolos por uart los valores de cada uno. \\ \hline
\end{tabular}
\caption{Caso de Uso - Puesta en alta del dispositivo}
\label{casos_uso_alta}
\end{table}


Si en el texto se hace alusión a diferentes partes del trabajo referirse a ellas como capítulo, sección o subsección según corresponda. Por ejemplo: ``En el capítulo \ref{Chapter1} se explica tal cosa'', o ``En la sección \ref{sec:ejemplo} se presenta lo que sea'', o ``En la subsección \ref{subsec:ejemplo} se discute otra cosa''.


\subsection{Requerimientos}
\label{subsec:ejemplo}

Se recomienda no utilizar \textbf{texto en negritas} en ningún párrafo, ni tampoco texto \underline{subrayado}. En cambio sí se sugiere utilizar \textit{texto en cursiva} donde se considere apropiado.

Se sugiere que la escritura sea impersonal. Por ejemplo, no utilizar ``el diseño del firmware lo hice de acuerdo con tal principio'', sino ``el firmware fue diseñado utilizando tal principio''. En lo posible hablar en tiempo pasado, ya que la memoria describe un trabajo que ya fue realizado.

Se recomienda no utilizar una sección de glosario sino colocar la descripción de las abreviaturas como parte del mismo cuerpo del texto. Por ejemplo, RTOS (\textit{Real Time Operating System}, Sistema Operativo de Tiempo Real) o en caso de considerarlo apropiado mediante notas a pie de página.

Si se desea indicar alguna página web utilizar el siguiente formato de referencias bibliográficas, dónde las referencias se detallan en la sección de bibliografía de la memoria,utilizado el formato establecido por IEEE en \citep{IEEE:citation}. Por ejemplo, ``el presente trabajo se basa en la plataforma EDU-CIAA-NXP, la cual se describe en detalle en \citep{CIAA}''.

\subsection{Planificación} 

La forma correcta de utilizar una figura es la siguiente: ``Se eligió utilizar un cuadrado azul para el logo, el cual se ilustra en la figura \ref{fig:cuadradoAzul}''.

\begin{figure}[h]
	\centering
	\includegraphics[scale=.35]{./Figures/cuadradoAzul.png}
	\caption{Ilustración del cuadrado azul que se eligió para el diseño del logo.}
	\label{fig:cuadradoAzul}
\end{figure}

El texto de las figuras debe estar siempre en español, excepto que se decida reproducir una figura original tomada de alguna referencia. En ese caso la referencia de la cual se tomó la figura debe ser indicada en el epígrafe de la figura e incluida como una nota al pie, como se ilustra en la figura \ref{fig:palabraIngles}.

\begin{figure}[h!]
	\centering
	\includegraphics[scale=.25]{./Figures/word.jpeg}
	\caption{Imagen tomada de la página oficial del procesador\protect\footnotemark.}
	\label{fig:palabraIngles}
\end{figure}

\footnotetext{\url{https://goo.gl/images/i7C70w}}


La figura y el epígrafe deben conformar una unidad cuyo significado principal pueda ser comprendido por el lector sin necesidad de leer el cuerpo central de la memoria. Para eso es necesario que el epígrafe sea todo lo detallado que corresponda y si en la figura se utilizan abreviaturas entonces aclarar su significado en el epígrafe o en la misma figura.


\subsection{Tablas}

Para las tablas utilizar el mismo formato que para las figuras, sólo que el epígrafe se debe colocar arriba de la tabla, como se ilustra en la tabla \ref{tab:peces}. Observar que sólo algunas filas van con líneas visibles y notar el uso de las negritas para los encabezados.  La referencia se logra utilizando el comando \verb|\ref{<label>}| donde label debe estar definida dentro del entorno de la tabla.

\begin{verbatim}
\begin{table}[h]
	\centering
	\caption[caption corto]{caption largo más descriptivo}
	\begin{tabular}{l c c}    
		\toprule
		\textbf{Especie}       & \textbf{Tamaño}  & \textbf{Valor aprox.}\\
		\midrule
		Amphiprion Ocellaris	  & 10 cm 			& \$ 6.000 \\		
		Hepatus Blue Tang      & 15 cm			 & \$ 7.000 \\
		Zebrasoma Xanthurus    & 12 cm			 & \$ 6.800 \\
		\bottomrule
		\hline
	\end{tabular}
	\label{tab:peces}
\end{table}
\end{verbatim}

\begin{table}[h]
	\centering
	\caption[caption corto]{caption largo más descriptivo}
	\begin{tabular}{l c c}    
		\toprule
		\textbf{Especie} 	 & \textbf{Tamaño}  & \textbf{Valor aprox.}  \\
		\midrule
		Amphiprion Ocellaris	 & 10 cm 			& \$ 6.000 \\		
		Hepatus Blue Tang	 & 15 cm				& \$ 7.000 \\
		Zebrasoma Xanthurus	 & 12 cm				& \$ 6.800 \\
		\bottomrule
		\hline
	\end{tabular}
	\label{tab:peces}
\end{table}

En cada capítulo se debe reiniciar el número de conteo de las figuras y las tablas, por ejemplo, Fig. 2.1 o Tabla 2.1, pero no se debe reiniciar el conteo en cada sección. Por suerte la plantilla se encarga de esto por nosotros.

\subsection{Ecuaciones}
\label{sec:Ecuaciones}

Al insertar ecuaciones en la memoria estas se deben numerar de la siguiente forma:

\begin{equation}
	\label{eq:metric}
	ds^2 = c^2 dt^2 \left( \frac{d\sigma^2}{1-k\sigma^2} + \sigma^2\left[ d\theta^2 + \sin^2\theta d\phi^2 \right] \right)
\end{equation}
                                                        
Es importante tener presente que en el caso de las ecuaciones estas pueden ser referidas por su número, como por ejemplo ``tal como describe la ecuación \ref{eq:metric}'', pero también es correcto utilizar los dos puntos, como por ejemplo ``la expresión matemática que describe este comportamiento es la siguiente:''

\begin{equation}
	\label{eq:schrodinger}
	\frac{\hbar^2}{2m}\nabla^2\Psi + V(\mathbf{r})\Psi = -i\hbar \frac{\partial\Psi}{\partial t}
\end{equation}

Para las ecuaciones se debe utilizar un tamaño de letra equivalente al utilizado para el texto del trabajo, en tipografía cursiva y preferentemente del tipo Times New Roman o similar. El espaciado antes y después de cada ecuación es de aproximadamente el doble que entre párrafos consecutivos del cuerpo principal del texto. Por suerte la plantilla se encarga de esto por nosotros.

Para generar la ecuación \ref{eq:metric} se utilizó el siguiente código:

\begin{verbatim}
\begin{equation}
	\label{eq:metric}
	ds^2 = c^2 dt^2 \left( \frac{d\sigma^2}{1-k\sigma^2} + 
	\sigma^2\left[ d\theta^2 + 
	\sin^2\theta d\phi^2 \right] \right)
\end{equation}
\end{verbatim}

Y para la ecuación \ref{eq:schrodinger}:

\begin{verbatim}
\begin{equation}
	\label{eq:schrodinger}
	\frac{\hbar^2}{2m}\nabla^2\Psi + V(\mathbf{r})\Psi = 
	-i\hbar \frac{\partial\Psi}{\partial t}
\end{equation}

\end{verbatim}