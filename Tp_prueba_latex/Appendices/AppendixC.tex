% Appendix Template

\chapter{ Requerimientos Funcionales y no Funcionales } % Main appendix title

\label{AppendixC} % Change X to a consecutive letter; for referencing this appendix elsewhere, use \ref{AppendixX}

\begin{enumerate}

\item Requerimientos Funcionales
\begin{enumerate}

\item Temperatura (RFTEM)
\begin{enumerate}
\item El sistema medirá la temperatura con un resolución de XX, cada YY segundos.
\item El sistema mantendrá la temperatura controlada según los parámetros configurados por el usuario.
\item El sistema elevará la temperatura a través de la activación de una salida digital conectada a una resistencia.
\item La activación de la resistencia se implementará a través de una ventana Smith Trigger para evitar la intermitencia y generación de ruido en las líneas de alimentación principal. Cruce por 0.
\item En caso de que la temperatura se exceda de rango de considerar interrumpir el proceso y emitir una alarma.
\item El sistema almacenará al menos XX valores de temperatura en un archivo en memoria flash.
\end{enumerate}

\item Energía (RFENE)
\begin{enumerate}
\item El software medirá la corriente total (DC) entregada al proceso de electrólisis cada XX segundos, con YY de resolución.
\item El software medirá la tensión aplicada (DC) entre los bornes del electrólisis cada XX segundos con YY de resolución.
\item El sistema almacenará al menos XX valores de corriente y tensión en un archivo en memoria flash.
\item Los rangos de valores óptimos de tensión y corriente serán tomados de los parámetros de lote ingresados por el usuario y mostrados por pantalla para su configuración manual en la fuente de alimentación principal.
\end{enumerate}

\item Conductividad (RFCOND)
\begin{enumerate}
\item El software calculará a través de la corriente y tensión medidas en el tanque de galvanización, la conductividad de la solución salina cada XX segundos con YY de resolución.
\item El software deberá compensar las eventuales desviaciones de la conductividad óptima a través de la activación de XX válvulas de aditivos.
\end{enumerate}

\item Interfaces Hombre-Máquina (RFHMI)
\begin{enumerate}
\item El sistema contará con una pantalla estado del sistema con variables a definir.
\item El sistema contará con un método de ingreso de parámetros de lote a procesar.
\item Deberá permitir ingresar parámetros en modo manual y otros en modo codificado.
\item Deberá brindar a través de una interfaz Ethernet los históricos almacenados en memoria flash de variables del proceso que necesiten ser auditadas tras una etapa o tras el proceso completo. El máximo de registros será de XX número de puntos en formato YY.
\end{enumerate}

\begin{enumerate}
\item Tiempos (RFTI)
\item El software llevará un conteo del tiempo entre cada baño en las bateas, desde el momento que se inicia hasta el final del proceso.
\item En cada etapa deberá avisar y esperar a que un operario habilite la iniciación de la siguiente etapa.
\end{enumerate}

\item Niveles de bateas (RFNB)	
\begin{enumerate}
\item Evaluará que los niveles dentro del galvanizador estén dentro de los rangos permitidos de operación.
\item En caso de algún nivel crítico se emitirán alarmas y se considera la interrupción del proceso.
\end{enumerate}
\end{enumerate}

\item Requerimientos de Interfaz
\begin{enumerate}

\item Temperatura (RITEM)
\begin{enumerate}
\item La temperatura será medida a través de un sensor analógico de tolerancia XX.
\item La temperatura se medirá en los tanques XX, lo que arroja un total de YY numero de entradas analógicas independientes.(+xxAI)
\end{enumerate}

\item Energía (RIENE)
\begin{enumerate}
\item Tendrá un sensor de alta corriente del tipo XX conectado en una entrada analógica, para la corriente de galvanizador. (+1AI)
\item Tendrá un sensor de tensión de tipo XX conectados a los bornes de galvanizador. (+1AI)
\end{enumerate}

\item Conductividad (RICOND)
\begin{enumerate}
\item Tendrá XX dispositivos dosificador/es conectados a salidas digitales. (+xxDO)
\end{enumerate}

\item Interfaces Hombre-Máquina (RIHMI)
\begin{enumerate}
\item Debe mostrar información a través de un puerto VGA/HDMI con una taza de refresco menor a XX segundos. (+1USB)
\item Tomará de una entrada serie USB los valores de lote. (+1USB)
\item Accionara a través de una salida digital una alarma sonora/lumínica en caso de algún tipo de falla. (+1DO, +1AO) 
\item Contará con uno o dos pulsadores a fin de poder detener y accionar el procesos de galvanización, conectados a una/dos entradas binarias.(+1/2DO)
\item Dispondrá de una conexión remota a través de Ethernet. (+1ETH)
\end{enumerate}

\item Niveles de bateas (RINB)
\begin{enumerate}
\item Tendrá XX sensores de nivel conectados a entradas digitales. (+xxDI)
\end{enumerate}

\end{enumerate}

\item Requerimientos no Funcionales (RNF)
\begin{enumerate}
\item Deberá ser probada la funcionalidad a través de un banco de pruebas que se ajuste al comportamiento del sistema. 
\end{enumerate}

\item Restricciones de Diseño (RD)
\begin{enumerate}
\item De los requerimientos de interfaz se resume que como mínimo el hardware deberá contar con las siguientes interfaces:
Entradas analógicas:(AI) >= 3	\\
Entradas digitales: (DI) >= 3	\\
Salidas analógicas: (AO) >= 1	\\
Salidas digitales:  (DO) >= xx	\\
Puerto USB:	  		(USB)  = 2	\\
Puerto RED:	  		(ETH)  = 1	\\
\end{enumerate}

\item Requerimientos a Futuro (RAF)
\begin{enumerate}
\item Brindar información acerca de si es necesario realizar una limpieza de sistema. Se puede utilizar como parámetro el número de procesos que se ejecutaron 	 	 	
\item Deberá permitir loguearse antes de iniciar el proceso, así tener un responsable de operación.
\item Deberá interactuar con una cinta de transportación automática que llevará las placas de una batea a otra. Accionara los motores (paso a paso?) de transporte y de elevación.
\item Si no se respetan los tiempos el sistema deberá dejar asentado el técnico y las acciones manuales ejecutadas a fin de tener un histórico antes posibles fallas en el lote.
\end{enumerate}
