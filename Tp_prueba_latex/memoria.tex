%%%%%%%%%%%%%%%%%%%%%%%%%%%%%%%%%%%%%%%%%
% Masters/Doctoral Thesis 
% LaTeX Template
% Version 2.3 (25/3/16)
%
% This template has been downloaded from:
% http://www.LaTeXTemplates.com
%
% Version 2.x major modifications by:
% Vel (vel@latextemplates.com)
%
% This template is based on a template by:
% Steve Gunn (http://users.ecs.soton.ac.uk/srg/softwaretools/document/templates/)
% Sunil Patel (http://www.sunilpatel.co.uk/thesis-template/)
%
% Template license:
% CC BY-NC-SA 3.0 (http://creativecommons.org/licenses/by-nc-sa/3.0/)
%
%%%%%%%%%%%%%%%%%%%%%%%%%%%%%%%%%%%%%%%%%

%----------------------------------------------------------------------------------------
%	PACKAGES AND OTHER DOCUMENT CONFIGURATIONS
%----------------------------------------------------------------------------------------
% The class file specifying the document structure:
\documentclass[
11pt, % The default document font size, options: 10pt, 11pt, 12pt
%oneside, % Two side (alternating margins) for binding by default, uncomment to switch to one side
%chapterinoneline,% Have the chapter title next to the number in one single line
%english, % ngerman for German
spanish,
singlespacing, % Single line spacing, alternatives: onehalfspacing or doublespacing
%draft, % Uncomment to enable draft mode (no pictures, no links, overfull hboxes indicated)
%nolistspacing, % If the document is onehalfspacing or doublespacing, uncomment this to set spacing in lists to single
%liststotoc, % Uncomment to add the list of figures/tables/etc to the table of contents
%toctotoc, % Uncomment to add the main table of contents to the table of contents
parskip, % Uncomment to add space between paragraphs
%nohyperref, % Uncomment to not load the hyperref package
headsepline, % Uncomment to get a line under the header
]{MastersDoctoralThesis} 

% Required for inputting international characters:
\usepackage[utf8]{inputenc}
% Output font encoding for international characters:
\usepackage[T1]{fontenc} 
% Use the Palatino font by default:
\usepackage{palatino} 
%,style=authoryear
% Use the bibtex backend with the authoryear citation style (which resembles APA)
\usepackage[backend=bibtex,natbib=true]{biblatex} 

% The filename of the bibliography:
\addbibresource{references.bib} 

% Required to generate language-dependent quotes in the bibliography:
\usepackage[autostyle=true]{csquotes} 

\usepackage{caption}
\usepackage{subcaption}

%------------------------
\usepackage{listings}
%\usepackage[hyphens]{url}
%\usepackage[hidelinks]{hyperref}
%\hypersetup{breaklinks=true}
\urlstyle{same}
%\usepackage{cite}
\usepackage{rotating}

%--------------------------
\usepackage{color}

%
%----------------------------------------------------------------------------------------
%	MARGIN SETTINGS
%----------------------------------------------------------------------------------------

\geometry{
	paper=a4paper, % Change to letterpaper for US letter
	inner=2cm, % Inner margin
	outer=3.3cm, % Outer margin
	bindingoffset=2cm, % Binding offset
	top=1.5cm, % Top margin
	bottom=1.5cm, % Bottom margin
	%showframe,% show how the type block is set on the page
}

%----------------------------------------------------------------------------------------
%	INFORMACIÓN DE LA MEMORIA
%----------------------------------------------------------------------------------------
% Comando: Referencia
\thesistitle{ Sistema de control para galvanoplastia de Circuitos Impresos } 
% \ttitle: El títulos de la memoria. 
\supervisor{ Ing. Juan Manuel Cruz (FIUBA,UTN-FRBA) } 
% \supname: El nombre del director. 
\degree{ Especialista en Sistemas Embebidos } 
% \degreename: Nombre del grado. 
\author{ Ing. Francisco Suárez } 
% \authorname: Tu nombre.
\juradoUNO{ Esp. Ing. Jorge Fonseca (FIUBA) } 
% \jur1name: Nombre y pertenencia del un jurado 1. 
\juradoDOS{ Mg. Ing. Diego Brengi (INTI, UNLaM, FIUBA) } 
% \jur2name: Nombre y pertenencia del un jurado 2.
\juradoTRES{ Esp. Ing. Ramiro Alonso (FIUBA) } 
% \jur3name: Nombre y pertenencia del un jurado 3. 
\fechaINICIO{ enero de 2017 }
% \fechaINICIOname:Fecha inicio proyecto.  
\fechaFINAL{ diciembre de 2018 }
% \fechaFINALname: Fecha finalacicion.

\subject{ Memoria del Trabajo Final de la Carrera de Especialización en Sistemas Embebidos de la UBA } 
% \subjectname: Your subject area, this is not currently used anywhere in the template.
\keywords{ CESE, Sistemas Embebidos, CIAA }
% \keywordnames: Keywords for your thesis, this is not currently used anywhere in the template.
\university{ Universidad de Buenos Aires } 
% \univname: Your university's name and URL, this is used in the title page and abstract.
\faculty{ {Facultad de Ingeniería} } 
% \facname: Your faculty's name and URL, this is used in the title page and abstract.
\department{ Departamento de Electrónica } 
% \deptname: Your department's name and URL, this is used in the title page and abstract.
\group{ {Laboratorio de Sistemas Embebidos} } 
% \groupname: Your research group's name and URL, this is used in the title page.
%  ^Estos comandos se usa en la carátula y se puede usar el cualquier lugar del documento.

% Set the PDF's title to your title:
\hypersetup{ pdftitle=\ttitle } 
% Set the PDF's author to your name:
\hypersetup{ pdfauthor=\authorname }
% Set the PDF's keywords to your keywords:
\hypersetup{ pdfkeywords=\keywordnames } 

\newcaptionname{spanish}{\acknowledgementname}{Agradecimientos}
\newcaptionname{spanish}{\authorshipname}{Declaración de Autoría}
\newcaptionname{spanish}{\abbrevname}{Glosario}
\newcaptionname{spanish}{\byname}{por}

% Listing -> Algorithm:
\renewcommand{\lstlistingname}{Algoritmo}
% List of Listings -> List of Algorithms:
\renewcommand{\lstlistlistingname}{Índice de \lstlistingname s}

\renewcommand{\listtablename}{Índice de Tablas}
\renewcommand{\tablename}{Tabla} 
 % Espacio adicional en los footnotes:
\addtolength{\footnotesep}{2mm}

\begin{document}
% Use roman page numbering style (i, ii, iii, iv...) for the pre-content pages:
\frontmatter 
% Default to the plain heading style until the thesis style is called for the body content:
\pagestyle{plain} 

%----------------------------------------------------------------------------------------
%	CARÁTULA
%----------------------------------------------------------------------------------------

\begin{titlepage}
\begin{center}

{\scshape\LARGE UNIVERSIDAD DE BUENOS AIRES\par}\vspace{0.1cm} % University name
{\scshape\LARGE FACULTAD DE INGENIERÍA\par}\vspace{0.1cm} % Faculty name
{\scshape\LARGE Carrera de Especialización en Sistemas Embebidos\par}\vspace{1cm} % Thesis type

\includegraphics[width=.3\textwidth]{./Figures/logoFIUBA.png}
\vspace{1cm}

\textsc{\Large Memoria del Trabajo Final}\\[0.5cm] % Thesis type

{\huge \bfseries \ttitle\par}\vspace{0.4cm} % Thesis title

\vspace{1cm}
\LARGE\textbf{Autor:\\
\authorname}\\ % Author name

\vspace{1cm}
\large
\vspace{10px}
{Director:} \\
{\supname} % Supervisor name
 
\vspace{1cm}
Jurados:\\
\jurunoname\\
\jurdosname\\
\jurtresname
 
\vfill
\textit{Este trabajo fue realizado en las Ciudad Autónoma de Buenos Aires, entre \fechaINICIOname \hspace{1px} y \fechaFINALname.}
\end{center}
\end{titlepage}


%----------------------------------------------------------------------------------------
%	RESUMEN - ABSTRACT 
%----------------------------------------------------------------------------------------

\begin{abstract}
\addchaptertocentry{\abstractname} % Add the abstract to the table of contents
%
%The Thesis Abstract is written here (and usually kept to just this page). The page is kept centered vertically so can expand into the blank space above the title too\ldots
\centering

Esta memoria presenta el desarrollo de un sistema de control y monitoreo de parámetros físico-químicos asociados a un proceso de galvanizado industrial, realizado para el fabricante de circuitos integrados Daichi SA. 

El sistema desarrollado asiste al personal técnico de planta en la optimización y mejora de los procesos vinculados al desarrollo de placas bicapa. A través de la lectura de sensores y almacenamiento de datos otorga funcionalidad similares a un PLC pero con funcionalidades a medida adicionales.

El prototipo se implementó sobre una EDU-CIAA y en el software se utilizaron técnicas de modelado orientado testing (TDD), y división por tareas para funcionar sobre un sistema multitareas. Además se aplicaron sistema control de versiones y trazabilidad de requerimientos.

\end{abstract}

%----------------------------------------------------------------------------------------
%	CONTENIDO DE LA MEMORIA  - AGRADECIMIENTOS
%----------------------------------------------------------------------------------------

\begin{acknowledgements}
%\addchaptertocentry{\acknowledgementname} % Descomentando esta línea se puede agregar los agradecimientos al índice
\vspace{1.5cm}

A Yesica mi compañera de la vida por la paciencia y a mi familia por el apoyo incondicional de siempre.

Especial agradecimiento al \supname por haberme iniciado el mundo de los sistemas embebidos. 

Y a todos los profesores de la CESE por el entusiasmo y las ganas que le ponen todos los días para que haya mejores profesionales.

 
\end{acknowledgements}

%----------------------------------------------------------------------------------------
%	LISTA DE CONTENIDOS/FIGURAS/TABLAS
%----------------------------------------------------------------------------------------
\renewcommand{\listtablename}{Índice de Tablas}

\tableofcontents % Prints the main table of contents

\listoffigures % Prints the list of figures

\listoftables % Prints the list of tables


%----------------------------------------------------------------------------------------
%	CONTENIDO DE LA MEMORIA  - DEDICATORIA
%----------------------------------------------------------------------------------------

\dedicatory{\textbf{Dedicado a... [OPCIONAL]}}  % escribir acá si se desea una dedicatoria

%----------------------------------------------------------------------------------------
%	CONTENIDO DE LA MEMORIA  - CAPÍTULOS
%----------------------------------------------------------------------------------------

\mainmatter % Begin numeric (1,2,3...) page numbering

\pagestyle{thesis} % Return the page headers back to the "thesis" style

\renewcommand{\tablename}{Tabla} 

% Incluir los capítulos como archivos separados desde la carpeta Chapters
% Descomentar las líneas a medida que se escriben los capítulos

% Chapter 1

\chapter{Introducción General} % Main chapter title

\label{Chapter1} % For referencing the chapter elsewhere, use \ref{Chapter1} 
\label{IntroGeneral}

%----------------------------------------------------------------------------------------

% Define some commands to keep the formatting separated from the content 
\newcommand{\keyword}[1]{\textbf{#1}}
\newcommand{\tabhead}[1]{\textbf{#1}}
\newcommand{\code}[1]{\texttt{#1}}
\newcommand{\file}[1]{\texttt{\bfseries#1}}
\newcommand{\option}[1]{\texttt{\itshape#1}}
\newcommand{\grados}{$^{\circ}$}

%----------------------------------------------------------------------------------------

%\section{Introducción}

%----------------------------------------------------------------------------------------
\section{Proceso de Galvanización en PCBs}

En el proceso de galvanización de pistas y vías de un PCBs el control de los parámetros físicos en cada una de las etapas es fundamental para garantizar la uniformidad del cobre. Este proceso consiste en la inmersión de los placas en distintos baños químicos que se observan en la Figura \ref{fig:thr_correcto}, y en la Figura \ref{fig:thr_correcto_perfil} el resultado de un correcto proceso donde el espesor del cobre conductor es uniforme en toda la cavidad.

\begin{figure}[h]
	\centering
	\includegraphics[width=.5\textwidth]{Figures/through_hole_correcto}
	\caption{Via con galvanización correcto en un PCB}
	\label{fig:thr_correcto}
\end{figure}

\begin{figure}[h]
\centering
\includegraphics[width=.5\textwidth]{Figures/through_hole_perfil}
\caption{Perfil de una via correctamente galvanizada}
	\label{fig:thr_correcto_perfil}
\end{figure}

Cuando este proceso no ocurre correctamente se originan distintas fallas, donde las más comunes son: vías sin galvanizar, vías obstruidas por exceso de cobre y capa no uniforme de metal cobre en la vía con riesgo de no conductividad. En la Figura \ref{thr_incorrecto_perfil} se observan las distintos grosores de cobre en vias mal galvanizadas en función de su ubicación en el sustrato del PCB.

\begin{figure}[h]
	\centering
	\includegraphics[width=.8\textwidth]{Figures/through_hole_perfil_fallado}
	\caption{Perfil un conjunto de sustratos con vías mal galvanizadas}
	\label{fig:thr_incorrecto_perfil}
\end{figure}

El proceso requiere una sucesión de baños por distintas soluciones químicas y enjuagues en agua, previas y después del proceso de electrolisis con cobre sobre el sustrato. En la Figura \ref{fig:diagrama_proceso} se resumen los pasos de forma simplificada, en el proceso de PCB se utilizan mas de 10 etapas.

\begin{figure}[h]
	\centering
	\includegraphics[width=.8\textwidth]{Figures/diagrama_galvanizado_castellano}
	\caption{Diagrama simplificado de una linea de galvanizado}
	\label{fig:diagrama_proceso}
\end{figure}


\LaTeX{} no es \textsc{WYSIWYG} (What You See is What You Get), a diferencia de los procesadores de texto como Microsoft Word o Pages de Apple o incluso LibreOffice en el mundo open-source. En lugar de ello, un documento escrito para \LaTeX{} es en realidad un archivo de texto simple, llano que \emph{no contiene formato} . Nosotros le decimos a \LaTeX{} cómo deseamos que se aplique el formato en el documento final escribiendo comandos simples entre el texto, por ejemplo, si quiero usar \emph{texto en cursiva para dar énfasis}, escribo \verb|\emph{texto}| y pongo el texto en cursiva que quiero entre medio de las llaves. Esto significa que \LaTeX{} es un lenguaje del tipo \enquote{mark-up}, muy parecido a HTML.

\section{Motivación y objetivo}

A partir de la búsqueda de incrementar los niveles de calidad en producto final surgió la necesidad de agregar control y monitoreo a determinadas bateas criticas en el proceso de modo de asistir a los operarios de planta en tiempo real sobre alguna anomalía en los parámetros antes de continuar con el proceso. En la Figura \ref{fig:planta_actual} se puede ver como es la linea de proceso actual.

\begin{figure}[h]
	\centering
	\includegraphics[width=.8\textwidth]{Figures/planta_actual}
	\caption{Linea de galvanizada actual operada manualmente}
	\label{fig:planta_actual}
\end{figure}

Como segundo objetivo como el proceso actual es totalmente hecho a mano y se esta trabajando en modernizarla y cambiarla de lugar se busca que el mismo sistema pueda integrar a una linea de traslación de los PCBs automatizada, lo cual se logró con el conteo de tiempo de posicionamiento por etapa.  

Finalmente a modo de control general y auditoría se necesitaba que el sistema registre los valores de las parámetros de medición en archivos . 

Si usted es nuevo a \LaTeX{}, hay un muy buen libro electrónico - disponible gratuitamente en Internet como un archivo PDF - llamado, \enquote{A (not so short) Introduction to \LaTeX{}}. El título del libro es generalmente acortado a simplemente \emph{lshort}. Puede descargar la versión más reciente en inglés (ya que se actualiza de vez en cuando) desde aquí:
\url{http://www.ctan.org/tex-archive/info/lshort/english/lshort.pdf}

Está disponible en varios idiomas además del inglés. Se puede encontrar la versión en español en la lista en esta página: \url{http://www.ctan.org/tex-archive/info/lshort/}


\section{Objetivo}

El presente sistema embebido se focalizó en la confiabilidad, robustez y adaptabilidad con las interfaces eléctricas necesarias para que funcione en un ambiente industrial con los actuadores y sensores determinados por el cliente. Para tal fin se propuso que el mismo funcione sobre la plataforma eduCIAA en conjunto con una placa de interfaz hecha a medida, como un prototipo de prueba previo al desarrollo de un hardware propio.

\subsection{Alcance}

El proyecto incluyó los siguientes puntos:
\begin{itemize}
	\item Estudio preliminar de las arquitecturas adecuadas para la implementación del sistema principal y subsistemas.
	\item Diseño de alto nivel (arquitectura) del sistema.
	\item Diseño del sistema en lenguaje C para plataforma CIAA.
	\item Plan de pruebas unitarias y ensayos (testbenchs) para cada subsistema.
	\item Plan de pruebas de integración y ensayos (testbenchs) para agrupaciones de subsistemas.
	\item Plan de pruebas del sistema y ensayos (testbenchs) para el sistema completo.
	\item Documentación del sistema y subsistemas que incluye:
	\begin{enumerate}
		\item Descripción de entradas y salidas (frecuencias, tamaño y tipos de datos, señales de control, etc.)
		\item Descripción de parámetros del sistema.
		\item Requerimientos funcionales implementados trazables a los requerimientos del proyecto (matriz de trazabilidad).
		\item Hipótesis de diseño, justificación de la elección del diseño, estudios previos y marco teórico.
		\item Diagrama de arquitectura.
		\item Reporte de ensayos realizados.
		\item Referencias bibliográficas.
	\end{enumerate}
	\item Análisis y construcción del banco de pruebas.
\end{itemize}

El proyecto no incluyó: 
\begin{itemize}
	\item Estudio de los sensores y actuadores, se basará dicha información en los datos dados por el cliente. 
	\item Análisis de mejor solución para implementación de sistema de reporte remoto de variables y registros históricos. 
	\item Test del sistema en lugar de producción. La planta se encontraba en reestructuración y modernización.
\end{itemize}



%----------------------------------------------------------------------------------------

\section{Utilizando esta plantilla}

Si usted está familiarizado con \LaTeX{}, entonces puede explorar la estructura de directorios de esta plantilla y proceder a personalizarla agregando su información en el bloque \emph{INFORMACIÓN DE LA PORTADA} en el archivo \file{memoria.tex}.  

Se puede continuar luego modificando el resto de los archivos siguiendo los lineamientos que se describen en la sección \ref{sec:FillingFile} en la página \pageref{sec:FillingFile}.

Asegúrese de leer el capítulo \ref{Chapter2} acerca de las convenciones utilizadas para las Memoria de los Trabajos Finales de la Carrera de Especialización en Sistemas Embebidos de FIUBA.

Si es nuevo en \LaTeX{} se recomienda que continue leyendo el documento ya que contiene información básica para aprovechar el potencial de esta herramienta.

Si usted está escribiendo un documento con mucho contenido matemático, entonces es posible que desee leer el documento de la AMS (American Mathematical Society) llamado, \enquote{A Short Math Guide for \LaTeX{}}. Se puede encontrar en línea en el siguiente link: \url{http://www.ams.org/tex/amslatex.html} en la sección \enquote{Additional Documentation} hacia la parte inferior de la página.



\subsection{Acerca de esta plantilla}

Esta plantilla \LaTeX{} está basada originalmente en torno a un archivo de estilo \LaTeX{} creado por Steve R.\ Gunn de la  University of Southampton (UK), department of Electronics and Computer Science. Se puede encontrar su trabajo original en el siguiente sitio de internet:
\url{http://www.ecs.soton.ac.uk/~srg/softwaretools/document/templates/}

El archivo de Gunn, \file{ecsthesis.cls} fue posteriormente modificado por Sunil Patel quien creó una plantilla esqueleto con la estructura de carpetas. El template resultante se puede encontrar en el sitio web de Sunil Patel:
\url{http://www.sunilpatel.co.uk/thesis-template}

El template de Patel se publicó a través de  \url{http://www.LaTeXTemplates.com} desde donde fue modificado muchas veces en base a solicitudes de usuarios. La versión 2.0 y subsiguientes representan cambios significativos respecto a la versión de la plantilla modificada por Patel, que es de hecho, dificilmente reconocible. El trabajo en la version 2.0 fue realizado por Vel Gayevskiy y Johannes Böttcher.

Uno de los primeros graduados de la Carrera de Especialización en Sistemas Embebios de la UBA, el Ing. \href{mailto:pbos@fi.uba.ar}{Patricio Bos} modificó los contenidos de la versión 2.3 para crear una plantilla altamente adaptada a la Carrera de Especialización de la UBA.

%----------------------------------------------------------------------------------------







\chapter{ Introducción Específica } % Main chapter title
%----------------------------------------------------------------------------------------
%	SECTION 1
%----------------------------------------------------------------------------------------
\section{ Que es el Galvanizado de un PCB? }

El proceso de galvanizado consiste en adherir a un objeto metálico una delgada capa de otro metal, de unos pocos micrones de espesor. Este material le otorga mejores propiedades al objeto como ser conductividad eléctrica, mayor resistencia mecánica o resistencia a la corrosión. 
Los métodos para lograrlos pueden ser diversos entre los mas comunes son con un baño químico y por electrolisis.
La forma utilizada en los PCBs (Printed Circuit Board o placa de circuito integrado) es a través de la electrolisis del objeto dentro de una solución con sales metálicas, mas una fuente continua de bajo voltaje y muy alta corriente, mas bloques del metal necesario para el galvanizado. 
La figura \ref{fig:galvanizado_electrolitico} es un diagrama simplificado de este proceso, aquí el ánodo es de cobre y el cátodo es el PCB a galvanizar.

\begin{figure}[h]
	\centering
	\includegraphics[width=.9\textwidth]{Figures/Cap_2/diagrama_galvanizado_basico}
	\caption{Proceso electrolítico con fuente continua.}
	\label{fig:galvanizado_electrolitico}
\end{figure}


\subsection{ Proceso de Galvanizado por Electrolisis }

En la fabricación de PCBs se parte de un material base de sustrato de laminas epoxi Fr4 revestido con laminas de cobre. Este debe ser sometido previo a realizar la electrolisis en si, a una sucesión de baños químicos para limpiar y homogeneizar toda las partes metálicas de superficie. El proceso es repetido con variantes del metal galvánico para lograr el circuito final. 
En la Figura \ref{fig:diagrama_proceso} se resumen los pasos de forma simplificada, en el proceso de PCB se utilizan mas de 10 etapas. 

\begin{figure}[h]
	\centering
	\includegraphics[width=1.0\textwidth]{Figures/Cap_2/diagrama_galvanizado_completo}
	\caption{Diagrama simplificado de las etapas.}
	\label{fig:diagrama_proceso}
\end{figure}

En este caso de estudio nos focalizamos en el primero que se usa tanto para metalizar las vías entre capas (previamente perforadas) como para engrosar el cobre base que originara las pistas del circuito. La figura \ref{fig:galvanizado_pcb} muestra el resultado final de un PTH (Plated Through Hole o agujero pasante metalizado) o simplemente vía, en un PCB.

\begin{figure}[h]
	\centering
	\includegraphics[width=.5\textwidth]{Figures/Cap_2/galvanizado_pcb}
	\caption{Vía entre capas galvanizada en un PCB}
	\label{fig:galvanizado_pcb}
\end{figure}

\section{ Parámetros críticos del galvanizado }

En el proceso de galvanización de PCBs el control de ciertos parámetros físicos-químicos en cada una de las etapas es fundamental para garantizar la uniformidad y calidad del baño de cobre. 

En las diferentes bateas de la linea de galvanizado es necesario controlar, según cada etapa uno o algunos de los siguientes parámetros físicos-químicos del proceso:
\begin{itemize}
	\item Temperatura.
	\item Nivel de líquidos.
	\item Conductividad o concentración de iones.
	\item Corriente entregada (electrolisis).
	\item Inyección de aire.
\end{itemize}

Actualmente en la linea de galvanizado los operadores solo cuentan con su experiencia y con el soporte de instrumental de medición básico, para usar en algunas las etapas de la linea. Para saber cuando una etapa esta realizada correctamente es necesario tener información precisa y detallada para garantizar el resultado. Esto es causante de que el producto final muchas veces no tenga una calidad estándar y que no se puedan prevenir fallas en proceso a tiempo.

\subsection{ Fallas en la galvanización de PCBs }

En la producción de PCBs el resultado final correcto es aquel donde el espesor del baño metálico es uniforme en toda la cavidad. Cuando este proceso no ocurre correctamente se originan distintas fallas, donde las más comunes son:
\begin{itemize}
	\item Vías sin galvanizar.
	\item vías obstruidas por exceso de cobre.
	\item Laminado no uniforme de metal cobre.
	\item Micro cortes del cobre y puntos de no conductividad.
\end{itemize}

En la Figura \ref{thr_incorrecto_perfil} se observan las distintos grosores de cobre con vías mal galvanizadas en función de su ubicación en el sustrato del PCB y la cercanía al ánodo de cobre.

\begin{figure}[h]
	\centering
	\includegraphics[width=.9\textwidth]{Figures/Cap_2/through_hole_perfil_fallado}
	\caption{Perfil de vías según la ubicación relativa en los sustratos y el los cátodos.}
	\label{fig:thr_incorrecto_perfil}
\end{figure}

En la figura \ref{thr_incorrecto_perfil} se observa los problemas originados por la mala preparación previa de las superficies metálicas del sustrato, originado puntos sin continuidad o circuitos abiertos que lo volverán no aptos para la producción de circuitos.

\begin{figure}[h]
	\centering
	\includegraphics[width=.5\textwidth]{Figures/Cap_2/etchback_thr_incorrecto_perfil}
	\caption{ Las capas de cobre se alejan del borde del orificio.}
	\label{fig:thr_incorrecto_perfil}
\end{figure}

\section{ Determinación de la solución } 

A fin de lograr determinar detalles de la arquitectura del sistema desarrollado es necesario definir y determinar casos de uso, requerimientos técnicos y definiciones generales originados por la nueva planta, su operación y las necesidades del cliente. 

\subsection{ Definiciones Generales }

Debido a que la linea cuenta con varias etapas en forma consecutiva independientes entre si pero con características similares, se determino implementar un prototipo capaz de procesar el conjunto de parámetros del proceso y capaz de interactuar con los distintos usuarios del entorno.

El sistema debería de funcionaria en modo predeterminado según los distintos escenarios posibles dentro de las etapas del proceso. De esta manera con un mismo hardware mas los sensores y actuadores necesarios, se puede hacer supervisión y control en las partes del proceso que se crean necesarias.

En una etapa posterior luego de la validación del prototipo el cliente desarrollaría un hardware a medida para usar en su planta. El mismo seria producido en las cantidades necesarias según el numero de etapas que se determinaran. Por ultimo estos distintos módulos deberían funcionar de manera independiente entre si.


\subsection{ Casos de Uso }

Se definieron los casos de uso del sistema para determinar como se vincularía el sistema con los distintos usuario, en función del modo de funcionamiento posible. Estos quedan entonces predeterminados por los escenarios posibles que se puede originar en la operación de la linea.
Los casos mas relevantes son:
\begin{enumerate}
	\item Puesta en alta y configuración del dispositivo, usado por personal técnico.
	\item Operación en modo de funcionamiento 1, usado por el operario y el técnico.
	\item Operación en modo de funcionamiento 2, usado por el operario y el técnico.
\end{enumerate}

En la Figuras \ref{fig:casoUsoAlta} y \ref{fig:casodeUso1y2} se notan las interacciones entre los distintos submódulos del sistema en función de los casos de usos y los usuarios vinculados.

\begin{figure}[h!]
	\centering
	\includegraphics[width=.7\textwidth]{Figures/Cap_2/caso_uso_Alta_UML}
	\caption{Diagrama UML caso de uso de puesta en alta.}
	\label{fig:casoUsoAlta}
\end{figure}

La puesta en alta es el momento en que se determinan a través de la interface terminal HMI (Humman Machine Interface o interfase hombre maquina) las parámetros a medir y configuraciones de los puertos del hardware.

\begin{figure}[h!]
	\centering
	\includegraphics[width=.7\textwidth]{Figures/Cap_2/caso_uso_1y2_UML}
	\caption{Diagrama UML de casos de uso Modo 1 y 2.}
	\label{fig:casodeUso1y2}
\end{figure}

Los modos 1 y 2 se refieren a dos configuraciones estandars y predefinidas que se crearon a partir de los escenarios mas comunes en las que se utilizaría el prototipo a fin de simplificar la puesta en marcha en campo.
En el apéndice \ref{AppendixA} se detallan la secuencias involucradas en cada una de los casos de uso.

\subsection{ Requerimientos Funcionales y no Funcionales }
\label{subsec:Requerimientos}

A fin de no ahondar con detalles técnicos el documento a lista completa con los requerimientos se pueden ver en el anexo \ref{AppendixC}. Cabe mencionar que el mismo se originó previamente en la materia Taller de Trabajo Final en el posgrado.
\\
En función de estos se propuso implementar el proyecto sobre una plataforma EDU-CIAA en conjunto con una placa de interfaz hecha a medida. 







 
\chapter{Diseño e Implementación} % Main chapter title

\label{Chapter3} % Change X to a consecutive number; for referencing this chapter elsewhere, use \ref{ChapterX}
\definecolor{mygreen}{rgb}{0,0.6,0}
\definecolor{mygray}{rgb}{0.5,0.5,0.5}
\definecolor{mymauve}{rgb}{0.58,0,0.82}

\lstset{ %
  backgroundcolor=\color{white},   % choose the background color; you must add \usepackage{color} or \usepackage{xcolor}
  basicstyle=\footnotesize,        % the size of the fonts that are used for the code
  breakatwhitespace=false,         % sets if automatic breaks should only happen at whitespace
  breaklines=true,                 % sets automatic line breaking
  captionpos=b,                    % sets the caption-position to bottom
  commentstyle=\color{mygreen},    % comment style
  deletekeywords={...},            % if you want to delete keywords from the given language
  %escapeinside={\%*}{*)},          % if you want to add LaTeX within your code
  %extendedchars=true,              % lets you use non-ASCII characters; for 8-bits encodings only, does not work with UTF-8
  %frame=single,	                   % adds a frame around the code
  keepspaces=true,                 % keeps spaces in text, useful for keeping indentation of code (possibly needs columns=flexible)
  keywordstyle=\color{blue},       % keyword style
  language=[ANSI]C,					% the language of the code
  %otherkeywords={*,...},           % if you want to add more keywords to the set
  numbers=left,                    % where to put the line-numbers; possible values are (none, left, right)
  numbersep=5pt,                   % how far the line-numbers are from the code
  numberstyle=\tiny\color{mygray}, % the style that is used for the line-numbers
  rulecolor=\color{black},         % if not set, the frame-color may be changed on line-breaks within not-black text (e.g. comments (green here))
  showspaces=false,                % show spaces everywhere adding particular underscores; it overrides 'showstringspaces'
  showstringspaces=false,          % underline spaces within strings only
  showtabs=false,                  % show tabs within strings adding particular underscores
  stepnumber=1,                    % the step between two line-numbers. If it's 1, each line will be numbered
  stringstyle=\color{mymauve},     % string literal style
  tabsize=2,	                   % sets default tabsize to 2 spaces
  title=\lstname,                   % show the filename of files included with \lstinputlisting; also try caption instead of title
  morecomment=[s]{/*}{*/}%
}


%----------------------------------------------------------------------------------------
%	SECTION 1
%----------------------------------------------------------------------------------------
\section{Análisis del Hardware}

El prototipo se implemento sobre una EDU-CIAA en conjunto con un hardware de adaptación de las interfaces. Para ello se utilizaron los proyectos de código abierto kicad \footnote{\url{http://kicad-pcb.org/download/}} para los ponchos de la eduCIAA \footnote{\url{https://github.com/brengi/Ponchos}} para adaptar las entradas y salidas del conectores de expansión con los sensores y actuadores utilizados. 

\subsection{Sensores y actuadores}

\subsubsection{Esquemáticos del prototipo final}
% Aca van todo lo referido a consideraciones importantes sobre los esquematicos.
Para la adaptación de las interfaces con termocuplas y termistores se necesitaron para el primer caso de un circuito integrado compensador de juntura debido a la alinealidad en la respuesta de ese tipo de sensores. Para ello se eligió el (MAX31855KASA+) como una solución simplificadora. También existen soluciones de amplificadores multietapas con compensaciones que resultaban mas económicos, tal como el que se puede observar en la Figura \ref{fig:cirCompTermocupla} que se adapto para la termocupla usada en el prototipo.

\begin{figure}[h!]
	\centering
	\includegraphics[width=.7\textwidth]{Figures/Cap_3/circuito_ampl_termocupla}
	\caption{Circuito simplificado de amplificación y compensación de termocuplas.} \protect\footnotemark
	\label{fig:cirCompTermocupla}
\end{figure}
\footnotetext{\url{http://ww1.microchip.com/downloads/en/AppNotes/00844a.pdf}}

Para el caso del termistor el circuito de adaptación fue mas simple debido a que señal obtenida es prácticamente lineal y solamente fue necesario amplificar a los niveles de trabajo de la entrada analógica del procesador. En la Figura XXXX se observa en tipo de circuito que se implementó.


\subsubsection{Renders del prototipo}

A continuación se muestran los renders de la placa de adaptación. En la Figuras \ref{fig:renderPonchoTOP} y \ref{fig:renderPonchoBOT} se ven las vistas del pcb prototipo propuesto a desarrollar para el sistema piloto. 

\begin{figure}[h!]
	\centering
	\includegraphics[width=.9\textwidth]{Figures/Cap_3/tempRelayPoncho_TOP}
	\caption{Render vista de arriba placa de interfaz.}
	\label{fig:renderPonchoTOP}
\end{figure}

\begin{figure}[h!]
	\centering
	\includegraphics[width=.9\textwidth]{Figures/Cap_3/tempRelayPoncho_BOTTOM}
	\caption{Render vista de abajo de placa de interfaz.}
	\label{fig:renderPonchoBOT}
\end{figure}


%----------------------------------------------------------------------------------------
%	SECTION 2
%----------------------------------------------------------------------------------------
\section{Análisis del software}
 
La idea de esta sección es resaltar los problemas encontrados, los criterios utilizados y la justificación de las decisiones que se hayan tomado.

Se puede agregar código o pseudocódigo dentro de un entorno lstlisting con el siguiente código:

\begin{verbatim}
\begin{lstlisting}[caption= "un epígrafe descriptivo"]

	las líneas de código irían aquí...
	
\end{lstlisting}
\end{verbatim}


\subsection{Diseño del Firmware}
\subsection{Capas de abstracción}
% Agregar imagen diagrama con un el sistema prototipo.
\subsection{Arquitectura del Software}

Patrón de Arquitectura
Capas: 
\begin{enumerate}
	\item Abstracción HAL.
	\item Sistema operativo.
	\item Aplicación.
\end{enumerate}

Estas estructura permitió trabajar con mayor abstracción del hardware según la cual el problema se divide en partes y los esfuerzos se concentro en la capa de Aplicación que es la integra la lógica del sistema mientras que las demás administran los recursos del microprocesador. A continuación se ofrece una descripción mayor de cada una:
\begin{enumerate}
	\item Se utiliza para lograr una abstracción del hardware. Se piensa a futuro poder migrar el software de plataforma según lo requiera la tecnificación del momento. En este caso se utiliza la librería LPC OPEN.

	\item Se utiliza un sistema operativo de tiempo real a fin de hacer un mejor uso del procesador y los recursos disponibles para cada una de las tareas que la aplicación demande. En este caso se utiliza el FREE RTOS.

	\item La aplicación se basa en el patrón Control Ambiental en donde el sistema incluye sensores que proporcionan información sobre el entorno y los actuadores que pueden cambiarlo. Este patrón se aplicara utilizando en distintos módulos o tareas funcionales sobre el sistema operativo.
\end{enumerate}

Aplicación
La capa esta compuesta por los siguientes módulos:
a- Interfaz de usuario (ingreso de parámetros, uart, lcd, alarmas) 
b- Control de temperatura
c- Monitoreo de nivel
d- Monitoreo de conductividad
e- Monitoreo de energía
f- Control de tiempos


\chapter{Ensayos y Resultados}
\label{Chapter4}

Para la validación del prototipo se realizaron distintas pruebas en hardware y el software a fin de cubrir los requerimientos explicados en \ref{subsec:Requerimientos}. Aquí se desarrollan las resultados para ambos partes de manera separada.

\section{ Pruebas funcionales del firmware }
\label{sec:pruebasFW}

El firmware esta basado en módulos propios mas módulos de terceros. En la figura \ref{fig:diag_Repositorio} se muestra la organización del repositorio \citep{firmwareTP} con los módulos de rutinas incluidos en cada uno. La estructura del firmware en si se basa en el repositorio utilizado en la materia de \textit{Sistemas Operativos de Tiempo Real} \citep{ws_ridolfi}.   
% Figura con arbol de directorios del proyecto
\begin{figure}[h!]
	\centering
	\includegraphics[width=1.0\textwidth]{Figures/Cap_4/diagrama_repositorio}
	\caption{Estructura de directorios del proyecto.}
	\label{fig:diag_Repositorio}
\end{figure}

En sub-rama \textit{support} se encuentran las rutinas de desarrollo propio, que consisten en:
\begin{itemize}
\item \textit{apiSupport}: rutinas API (Application Programming Interface) interfaz de programación de aplicaciones.
\item \textit{ciaaSupport}: rutinas de soporte del hardware propio de la EDU-CIAA.
\item \textit{dspMath}: rutinas de soporte matemático a operaciones de procesado y conversión de señales físicas.
\item \textit{taskServices}: rutinas principales de los servicios o tareas del programa.
\end{itemize}

El directorio \textit{LPC4337} básicamente se componen de las librerías del fabricante denominada LPC-OPEN \citep{lpcopen}. Allí se encuentran los subdirectorios \textit{chip} y \textit{base} donde se alojan los módulos de capa HAL \footnotemark y capa CMSIS \footnotemark respectivamente.\\
En la ubicación \textit{External} se encuentran los módulos con librerías de terceros. En principio se muestran solo los que se usaron, los que no se dejan para futuras implementaciones. Se utilizaron los siguientes módulos:
\begin{itemize}
\item \textit{freeRTOS}: sistema operativo de tiempo real \citep{free_rtos}.
\item \textit{ntShell}: natural tiny Shell \citep{nt_shell}.
\item \textit{dsp}: rutinas de filtrado digital de señales.
\end{itemize}


\subsection{ Pruebas unitarias }

Se realizaron sobre los módulos propios ubicados en el directorio \textit{Support} del repositorio. Los mismos interactúan entre los servicios de la aplicación y las rutinas de capa HAL. Sobre la capa HAL y los librerías de terceros no se realizaron pruebas, se confió en el funcionamientos de los mismos.

A fin de validar el comportamiento de los algoritmos de procesamiento de los sensores de temperatura se realizaron pruebas sobre archivos con datos de entradas y se generaron archivos de salida con los resultados obtenidos. En la figura \ref{fig:diag_test_temp} se observa los tipos de archivos que se procesan.
% RESULTADOS DE LOS TEST DE TEMPERATURA
\begin{figure}[h!]
	\centering
	\includegraphics[width=1.0\textwidth]{Figures/Cap_4/diag_test_temperatura}
	\caption{Test de funciones interfaz de temperatura.}
	\label{fig:diag_test_temp}
\end{figure}

\footnotetext{HAL: Hardware Abstraction Level.}
\footnotetext{CMSIS: Cortex Microcontroller Software Interface Standard.}

La escritura en memoria interna se probó un procedimiento de emulación de datos con archivos de texto plano de entrada y salida,  validando la escritura de bloques binarios según los formatos establecidos en la lógica del programa. En la figura \ref{fig:diag_test_mem} se observa los archivos procesados en el test.
%RESULTADOS DE LOS TEST DE ESCRITURA EN MEMORIA
\begin{figure}[h!]
	\centering
	\includegraphics[width=0.9\textwidth]{Figures/Cap_4/diag_test_memoria}
	\caption{Test de funciones interfaz de memoria.}
	\label{fig:diag_test_mem}
\end{figure}

Para estas pruebas se predefinieron los archivos de ingreso de datos de manera de simplificar la lectura de los bloques de información. En las figuras \ref{fig:files_test_mem} y \ref{fig:files_test_temp} se observan la estructura de los archivos procesados.

\begin{figure}[h!]
	\centering
	\includegraphics[width=0.9\textwidth]{Figures/Cap_4/files_test_temperatura}
	\caption{ Archivos procesados en el test de temperatura.}
	\label{fig:files_test_temp}
\end{figure}

\begin{figure}[h!]
	\centering
	\includegraphics[width=0.9\textwidth]{Figures/Cap_4/files_test_memoria}
	\caption{Archivos procesados en el test de memoria.}
	\label{fig:files_test_mem}
\end{figure}


\section{ Ensayos de validación hardware }
\label{sec:pruebasHW}

Para esto se realizaron las pruebas sobre los puertos de la EDU-CIAA emulando el comportamiento de los sensores en función de los modos de funcionamiento preestablecidos. En la figura \ref{fig:foto_prototipo} se observa el banco de pruebas que emula las entradas y salidas que se tendrían en el entorno real.
% AGREGAR CAPTURAS DEL SISTEMA - BANCO DE PRUEBA CON LOS RESULTADOS.
% figuras de modificion de estructuras y pantalla?
\begin{figure}[h!]
	\centering
	\includegraphics[width=1.0\textwidth]{Figures/Cap_4/foto_prototipo}
	\caption{Banco de pruebas.}
	\label{fig:foto_prototipo}
\end{figure}

\subsection{ Pruebas de interfaz }

Para esto simplemente se ingresaron los comandos establecidos y verificaron que los mismos ejecuten correctamente las acciones internas así también como los mensajes por la terminal.Se tomaron luego capturas de los estados de salida así como de la terminal y de log registrado por el sistema. En las figuras \ref{falta_imagen} se muestran las ejecuciones de algunos de ellos.

Se probó el funcionamiento de los comandos vía consola y su impacto sobre las configuraciones de los puertos y la devolución de estados por pantalla. 

% figuras de modificion de estructuras y pantalla?
\begin{figure}[h!]
	\centering
	\includegraphics[width=1.0\textwidth]{Figures/Cap_4/captura_consola}
	\caption{Lista de comando implementados.}
	\label{fig:term_Configuracion}
\end{figure}


\section{ Ensayos de integración }
\label{sec:pruebasINT}

A fin de validar que el sistema completo, se realizaron pruebas de cada una de las partes del prototipo. Se probaron utilizando el módulo externo de simulación de entradas y salidas mas la terminal serial de pc conectado. A partir de la pre-configuración de las puertos del prototipo realizando variaciones en las entradas y verificando las salidas y los mensajes por la terminal.


\subsection{ Validación de requerimientos }

Para esto se tomaron los criterios desarrollados conjuntamente con los requerimientos previamente en la materia \emph{Gestión de Proyectos} del posgrado. Básicamente consiste en variar los valores de los entradas al prototipo y verificar que cumpla con la respuesta adecuada. En la tabla \ref{tablas_cumplimiento_req} se observa un resumen del cumplimiento de los requerimientos funcionales y de interfaz. Para los mismos se evaluó si cumple si o no, y en caso negativo porque no.
% tabla con los resultados de cumplimiento.
\begin{table}[h!]
\begin{flushleft}
\begin{tabular}{|m{2.6cm}|m{1.5cm}|m{1.5cm}|m{6.8cm}|}\hline
{\textbf{Requerimiento}} & {\textbf{Número}} & {\textbf{Cumple}} & {\textbf{Observaciones}}\\ \hline
\multicolumn{1}{|l|}{RFTEM} & { 1.1.1 } & { SI } & { Ninguna. }\\ \cline{ 2- 4}
\multicolumn{1}{|l|}{} & { 1.1.2 } & { SI } & { Ninguna. } \\ \cline{ 2- 4}
\multicolumn{1}{|l|}{} & { 1.1.3 } & { SI } & { Ninguna. } \\ \cline{ 2- 4}
\multicolumn{1}{|l|}{} & { 1.1.4 } & { SI } & { Ninguna. } \\ \cline{ 2- 4}
\multicolumn{1}{|l|}{} & { 1.1.5 } & { SI } & { Ninguna. } \\ \cline{ 2- 4}
\multicolumn{1}{|l|}{} & { 1.1.6 } & { NO } & { Memoria interna insuficiente. } \\ \hline
\multicolumn{1}{|l|}{RFENE} & { 1.2.1 } & { SI } & { Ninguna. }\\ \cline{ 2- 4} 
\multicolumn{1}{|l|}{} & { 1.2.2 } & { NO } & { Se supone constante 5V. } \\ \cline{ 2- 4}
\multicolumn{1}{|l|}{} & { 1.2.3 } & { NO } & { Memoria interna insuficiente. } \\ \cline{ 2- 4}
\multicolumn{1}{|l|}{} & { 1.2.4 } & { SI } & { Por ahora solo se pueden ingresar comandos. } \\ \hline
\multicolumn{1}{|l|}{RFCOND} & { 1.3.1 } & { SI } & { Ninguna. }\\ \cline{ 2- 4} 
\multicolumn{1}{|l|}{} & { 1.3.2 } & { SI } & { Ninguna. }\\ \hline
\multicolumn{1}{|l|}{RFTI} & { 1.4.1 } & { NO } & { No implementado sistema distribuido. }\\ \cline{ 2- 4} 
\multicolumn{1}{|l|}{} & { 1.4.2 } & { SI } & { Ninguna. }\\ \cline{ 2- 4} 
\multicolumn{1}{|l|}{} & { 1.2.3 } & { NO } & { No implementado sistema distribuido. }\\ \hline
\multicolumn{1}{|l|}{RFNB} & { 1.5.1 } & { SI } & { Ninguna. }\\ \cline{ 2- 4} 
\multicolumn{1}{|l|}{} & { 1.5.2 } & { SI } & { Ninguna. }\\ \hline
\multicolumn{1}{|l|}{RFHMI} & { 1.6.1 } & { SI } & { Ninguna. }\\ \cline{ 2- 4} 
\multicolumn{1}{|l|}{} & { 1.6.2 } & { NO } & { No implementado sistema distribuido. }\\ \cline{ 2- 4} 
\multicolumn{1}{|l|}{} & { 1.6.3 } & { SI } & { Ninguna. }\\ \cline{ 2- 4} 
\multicolumn{1}{|l|}{} & { 1.6.4 } & { NO } & { No implementado sistema distribuido. }\\ \cline{ 2- 4} 
\multicolumn{1}{|l|}{} & { 1.6.5 } & { NO } & { No implementado sistema distribuido. }\\ \hline 
\multicolumn{1}{|l|}{RITEM} & { 2.1.1 } & { SI } & { Ver sección \ref{analisis_hardware}. }\\ \cline{ 2- 4} 
\multicolumn{1}{|l|}{} & { 2.1.2 } & { SI } & { Ver sección \ref{analisis_hardware}. }\\ \hline 
\multicolumn{1}{|l|}{RIENE} & { 2.2.1 } & { SI } & { Ninguna. }\\ \cline{ 2- 4} 
\multicolumn{1}{|l|}{} & { 2.2.2 } & { NO } & { Se supone parámetro constante 5V. }\\ \hline
\multicolumn{1}{|l|}{RICOND} & { 2.3.1 } & { SI } & { Ninguna. }\\ \cline{ 2- 4} 
\multicolumn{1}{|l|}{} & { 2.3.2 } & { SI } & { Ninguna. }\\ \hline
\multicolumn{1}{|l|}{RINB} & { 2.4.1 } & { SI } & { Ninguna. }\\ \hline
\multicolumn{1}{|l|}{RIHMI} & { 2.5.1 } & { SI } & { Ninguna. }\\ \cline{ 2- 4} 
\multicolumn{1}{|l|}{} & { 2.5.2 } & { SI } & { Solo con comandos. }\\ \cline{ 2- 4} 
\multicolumn{1}{|l|}{} & { 2.5.3 } & { SI } & { Ninguna. }\\ \cline{ 2- 4} 
\multicolumn{1}{|l|}{} & { 2.5.4 } & { SI } & { Ninguna. }\\ \cline{ 2- 4} 
\multicolumn{1}{|l|}{} & { 2.5.5 } & { SI } & { Ninguna. }\\ \cline{ 2- 4} 
\multicolumn{1}{|l|}{} & { 2.5.6 } & { NO } & { Se puede implementar en la misma plataforma a futuro. }\\ \hline
\multicolumn{1}{|l|}{RNF} & { 3.1 } & { NO } & { No se construyó banco de pruebas. }\\ \hline 
\multicolumn{1}{|l|}{RD} & { 4.1 } & { SI } & { Ver sección \ref{analisis_hardware}.}\\ \hline 
\end{tabular}
\end{flushleft}
\caption{ Cumplimiento de los requerimientos.}
\label{tablas_cumplimiento_req}
\end{table}

Los requerimientos a Futuro (RAF) no se han agregado a la tabla pero fueron considerados durante el desarrollo del prototipo como un sistema distribuido. Esa arquitectura fue descartada y dejada para mas adelante en función de los resultados obtenidos con éste primer diseño.\\
A su vez muchos de los requerimientos originales estaban asociados a realizar un sistema distribuido de control con módulos de entradas y salidas interconectados a la EDU-CIAA. Si bien ese proyecto fue modificado los requerimiento se conservaron a modo ilustrativo de como se transformó el proyecto desde su idea inicial.
\subsection{ Pruebas de rendimiento }

Se probó la estabilidad del mismo dejando en funcionamiento durante un día completo, de esta manera se tiene mejor información de la perfomance de tiempo de uso y memoria de los servicios en el sistema operativo. En la tabla \ref{tabla_rendimiento} se observa el resultado obtenido en uso de memoria y tiempo de CPU.

% Tabla con los valores de rendimiento
\begin{table}[h!]
\centering
\begin{tabular}{ m{1.0cm}|m{3.0cm}|m{2.0cm}|m{2.5cm} }\hline
{\textbf{Id}} &{\textbf{Nombre}} & {\textbf{CPU(\%)}} & {\textbf{Memoria(Byte)}}\\ \hline
{0} & {Task Idle} & { 95 } & { 128 }\\ \hline
{1} & {Task Outputs} & { 1 } & { 256 }\\ \hline
{2} & {Task Inputs}  & { 1 } & { 256 }\\ \hline
{3} & {Task Terminal} & { 3 } & { 512 }\\ \hline
{4} & {Task Eeprom } & { 1 } & { 256 }\\ \hline
\end{tabular}
\caption{Uso de recursos por tarea.}
\label{tabla_rendimiento}
\end{table}

Considerando que este microprocesador (LPC-4337) tiene 32KB de memoria RAM, con mas la mitad disponible las tareas solo consumen 1408 Bytes y menos del 4\% del CPU. Por lo tanto se tienen recursos de sobra para poder seguir implementando servicios y funciones en la plataforma.

\chapter{ Conclusiones }
\label{Chapter5} 
%----------------------------------------------------------------------------------------
En este ultimo capitulo se vuelcan las conclusiones acerca del prototipo y la funcionalidad lograda. También se analizan si las opciones en el proceso de desarrollo fueron las correctas. Finalmente se describen los cambios y mejoras que se le realizaran a futuro a fin de mejorar las funcionalidades y optimizar la mantenibilidad del mismo. 

%----------------------------------------------------------------------------------------
%	SECTION 1
%----------------------------------------------------------------------------------------
\section{ Conclusiones generales }

Se describen las conclusiones separando los entornos de hardware y de software. 

\subsection{ Hardware del prototipo }
A nivel particular de la implementación realizada se logro cumplir con gran parte de los requisitos originales y con la mayoría de los requisitos originados para el prototipo final. 
El sistema originalmente se pensó para ser implementado sobre un microprocesador con menores prestaciones que el del micro de la EDU-CIAA. Como sobre éste demostró estar holgado de recursos, se concluye que la transición hacia una plataforma mas reducida es realizable.

Los elementos seleccionado como interfaz de hardware demostraron ser suficientes para integrar en una placa dedicada, de esta manera se pueden reducir los costos al no usar módulos con soluciones de terceros que muchas veces elevan altamente los costos de los sensores de campo.

\subsection{ Firmware del prototipo }
El presente sistema desarrollado sobre una necesidad especifica permitió lograr un sistema base de control y monitoreo de parámetros físicos en un proceso de fabricación. El firmware a su vez cuenta con los siguientes características mas sobresalientes:
\begin{itemize}
\item Es un sistema simple y modularizado en tareas independientes, lo cual lo hace mantenible y testeable.
\item Los procesos que funcionan sobre el sistema operativo están pensado para ser compilados de forma opcional.
\item Se pueden incrementar los servicios o quitarlos en función de modificar su funcionalidad.
\item Tiene gran disponibilidad de tiempo de procesador y de memoria para agregarle funcionalidad.
\item Es portable hacia otras plataformas de LPC M4 con solo modificar las librerías lpc open.
\end{itemize}

\subsection{ Proceso de desarrollo }

A nivel del proceso de desarrollo se pueden definir una serie de hitos positivos y otro negativos. Como positivos están:
\begin{itemize}
\item Ajustarse a la estructura del proceso de análisis de sistema, casos de uso y requerimientos permitió una trazabilidad entre las características del prototipo.
\item Se logró un producto mínimo viable óptimo.
\item Utilizar un sistema de control de versiones (GIT) del firmware permitió tener un seguimiento claro de las funcionalidades logradas durante el proceso.
\item Diseñar el firmware en módulos independientes permitió lograr un análisis de calidad del mismo de manera mas simple y confiable.
\end{itemize}

Como rasgos negativos se pueden destacar:
\begin{itemize}
\item Dedicarle extenso tiempo hasta la definiciones especificas de los requerimientos resultó contraproducente por las demoras para iniciar con el proyecto.
\item Implementar los módulos de interfaces con sensores de temperatura implicó mas tiempo del pensado. 
\item No haber desarrollado una arquitectura completa del sistema antes de iniciarlo produjo mas demoras que si se hubiera modelizado integro desde un principio.
\end{itemize}


%----------------------------------------------------------------------------------------
%	SECTION 2
%----------------------------------------------------------------------------------------
\section{ Próximos pasos }

A nivel de software quedaron varios puntos abiertos para darle mas prestaciones. Entre ellos se implementaran:
\begin{itemize}
\item Agregar mas comandos sobre la comunicación vía terminal-UART para tener mas prestaciones. 
\item Desarrollar servicios de comunicación vía I2C para interactuar con módulos externos, particularmente sensores.
\item Desarrollar servicios de comunicación MODBUS-RS485 para interactuar con otros plataformas.
\item Implementar un módulo de servicios UDP-Ethernet para pode comunicarse a futuro con un sistema central tipo SCADA.
\item Mejorar los módulos desarrollados estandarizando mas las nombres de las rutinas y reestructurando archivos headers.
\item Agregar mas protocolos de prueba y validación para poder agregarlos a futuro a un servidor de verificación tipo Jenkins.
\end{itemize}

A nivel de hardware se piensan a futuro los siguientes pasos:
\begin{itemize}
\item Probar la portabilidad de sistema hacia una plataforma con ARM M4 de prestaciones reducidas.
\item Elaborar el "poncho" de la EDU-CIAA como se pensó originalmente pero agregándole integrado mas acordes a los que se pueden conseguir en el mercado local.
\end{itemize}


%----------------------------------------------------------------------------------------
%	CONTENIDO DE LA MEMORIA  - APÉNDICES
%----------------------------------------------------------------------------------------

\appendix % indicativo para indicarle a LaTeX los siguientes "capítulos" son apéndices

% Incluir los apéndices de la memoria como archivos separadas desde la carpeta Appendices
% Descomentar las líneas a medida que se escriben los apéndices

% Appendix A

\chapter{Casos de Uso} % Main appendix title

\label{AppendixA} % For referencing this appendix elsewhere, use \ref{AppendixA}

Referencias:

Ref 1-  Disparadores:  Evento comienza el caso de uso.\\
Ref 2 - Flujo básico:  Pasos del escenario, desde que es disparado hasta que alcanza su objetivo.\\
Ref 3 - Flujo alternativo:  Pasos alternativos al flujo básico. \\
Ref 4-  Pre-Condiciones:  Condiciones que deben estar presentes para que se pueda iniciar el caso de uso.\\
Ref 5-  Pos-Condiciones: Condiciones que deben estar presentes para que se pueda finalizar el caso de uso.\\
Ref 6-  Modo 1: Medición de conductividad, nivel y control de temperatura.\\
Ref 7-  Modo 2: Medición de corriente, tensión, nivel y control de temperatura.\\

% Tablas con los casos de Uso --------------------------------------------------------------------%
% Puesta en alta del dispositivo:
\begin{table}[ht]
\begin{flushleft}
\begin{tabular}{|m{3.2cm}|m{11cm}|}\hline
\multicolumn{1}{|c|}{\textbf{Título}} & \multicolumn{1}{c|}{\textbf{Descripción}} \\ \hline
1. Nombre & Puesta en alta del dispositivo. \\ \hline
1.1. Descripción  & Secuencia de instalación de dispositivo en las cubas de control. \\ \hline
1.2. Actor principal  & Técnico instalador. \\ \hline
1.3. Disparadores  & Alimentación de la unidad. \\ \hline
2. Flujo de eventos &  \\ \hline
\multicolumn{ 1}{|l|}{2.1. Flujo básico} & 2.1.1. El sistema inicia el ciclo de encendido y se prepara para evaluar los rangos de las entradas de una por vez. 2.1.4  Censa que los  de los termostatos sean los correctos. (a definir) \\ \cline{ 2- 2}
\multicolumn{ 1}{|l|}{} & 2.1.2. Para cada uno emite una señal a definir en caso de ser correcto y otra en caso de ser incorrecto.  \\ \cline{ 2- 2}
\multicolumn{ 1}{|l|}{} & 2.1.3. Censa que los valores de corriente sobre los lazos de control de conductividad sean los correctos, según el máximo y el mínimo admisible.(a definir)  \\ \cline{ 2- 2}
\multicolumn{ 1}{|l|}{} & 2.1.4. Censa que los valores de los termostatos sean los correctos. (a definir) \\ \cline{ 2- 2}
\multicolumn{ 1}{|l|}{} & 2.1.5. Censa los valores de las entrada de nivel, que deben estar dentro de los establecidos según RINB242? \\ \hline
2.2. Flujo alternativo  & 2.1.6. Si alguno de los sensores no esta dentro de los valores se bloquea el proceso hasta que se seleccione ignorar o se normalice el parámetro.  \\ \hline
3. Requerimientos especiales & Botón de encendido liberado. Selector de modo de funcionamiento en posición. \\ \hline
4. Pre Condiciones  & Interfaces de sensores conectados.  \\ \hline
5. Pos Condiciones & El dispositivo queda en modo espera a la señal de inicio de ciclo, midiendo según los requerimientos RFTEM112,121,131,153? y mostrándolos por uart los valores de cada uno. \\ \hline
\end{tabular}
\end{flushleft}
\caption{Descripción caso de uso puesta en alta}
\label{caso_uso_alta}
\end{table}

% Secuencia de pasos para configurar dispositivo en el modo 1 de uso:
\begin{table}[h!]
\begin{flushleft}
\begin{tabular}{|m{3cm}|m{11cm}|}\hline
\multicolumn{1}{|c|}{\textbf{Título}} & \multicolumn{1}{c|}{\textbf{Descripción}} \\ \hline
1. Nombre & Puesta en funcionamiento en Modo 1 \\ \hline
1.1 Breve descripción  & Secuencia de control de un ciclo completo dentro de las cuba controlada  \\ \hline
1.2 Actor principal  & Operario (Op) \\ \hline
1.3 Disparadores  & Pulsación de botón inicio \\ \hline
2. Flujo de eventos &  \\ \hline
\multicolumn{ 1}{|l|}{2.1. Flujo básico} & 2.1.1. El sistema detecta la presencia de las placas con el detector colocado en la cuba. \\ \cline{ 2- 2}
\multicolumn{ 1}{|l|}{} & 2.1.2. El sistema inicia el conteo del tiempo preestablecido para esa etapa utilizando un indicar luminoso para indicar al Op. \\ \cline{ 2- 2}
\multicolumn{ 1}{|l|}{} & 2.1.3. El sistema continua controlando los parámetros de temperatura y nivel durante todo el ciclo y transmitiendo los valores por uart. \\ \cline{ 2- 2}
\multicolumn{ 1}{|l|}{} & 2.1.4. Completado el ciclo el sistema emite una señal al Op para que se disponga a retirar las placas de la cuba. \\ \hline
\multicolumn{ 1}{|l|}{2.2. Flujo alternativo } & 2.2.1. El sistema no detecta la placa con los sensores de presencia y no inicia el conteo de tiempo. \\ \cline{ 2- 2}
\multicolumn{ 1}{|l|}{} & 2.2.2. Si se colocan las placas o se anula la entrada del sensor el proceso vuelve a 2.1.2 del flujo básico. \\ \hline
\multicolumn{ 1}{|l|}{2.3. Flujo alternativo } & 2.3.1. El Op retira las placas de la cuba antes de completado el tiempo reglamentario. El sistema emite una seña de alarma para que se restablezca la placa durante 15 segundos. \\ \cline{ 2- 2}
\multicolumn{ 1}{|l|}{} & 2.3.2. Si se restablece la placa antes de cumplirse los 15 segundos se retorna al paso normal en el que se encontraba. Sino vuelve el contador de tiempo a cero y deja un indicador de alarma encendido hasta que se lo anule manualmente y vuelve al punto 2.1.5 del flujo básico. \\ \hline
\multicolumn{ 1}{|l|}{2.4. Flujo alternativo } & 2.4.1. El sistema detecta que la temperatura o el nivel fuera de valor nominal por mas de 5 minutos emite una alarma y continua el proceso normal hasta 2.1.5.\\ \cline{ 2- 2}
\multicolumn{ 1}{|l|}{} & 2.4.2.  Finalizado el tiempo no permite iniciar un nuevo ciclo hasta que las parámetros de control vuelvan a sus valores nominales.  \\ \hline
3. Requerimientos especiales & El sistema debe estar configurado en el Modo 1. \\ \hline
4. Pre condiciones  &  Sistema energizado e inicializado.   \\ \hline
5. Pos condiciones &  Queda en modo espera controlando los parámetros según los requerimientos RFTEM112,121,131,153? \\ \hline
\end{tabular}
\end{flushleft}
\caption{Descripción caso de uso en Modo 1}
\label{caso_uso_func_1}
\end{table}

% Secuencia de pasos para configurar dispositivo en el modo 2 de uso:

\begin{table}[h!]
\begin{flushleft}
\begin{tabular}{|m{3cm}|m{11cm}|}
\hline
\multicolumn{1}{|c|}{\textbf{Título}} & \multicolumn{1}{c|}{\textbf{Descripción}} \\ \hline
1. Nombre & Puesta en funcionamiento en Modo 2. \\ \hline
1.1 Breve descripción  & Secuencia de control de un ciclo completo dentro de las cuba controlada. \\ \hline
1.2 Actor principal  & Operario (Op) \\ \hline
1.3 Disparadores  & Pulsación de botón inicio. \\ \hline
2. Flujo de eventos &  \\ \hline
\multicolumn{ 1}{|l|}{2.1 Flujo básico} & 2.1.1. El sistema detecta la presencia de las placas con el detector colocado en la cuba.2.1.2  El sistema detecta la  umbral y comienza a integrar el valor hasta llegar al total necesario.2.1.4  Completado el ciclo el sistema emite una señal al Op para que se disponga a retirar las placas de la cuba. \\ \cline{ 2- 2}
\multicolumn{ 1}{|l|}{} & 2.1.2. El sistema detecta la corriente umbral y comienza a integrar el valor hasta llegar al total necesario. \\ \cline{ 2- 2}
\multicolumn{ 1}{|l|}{} & 2.1.3. El sistema inicia un indicar luminoso para indicar al Op que la etapa esta en proceso. \\ \hline
\multicolumn{ 1}{|l|}{2.2. Flujo alternativo} & 2.2.1. El valor de temperatura se va de rango por mas de 10 minutos, el sistema emite una alarma sin interrumpir el proceso. \\ \hline
\multicolumn{ 1}{|l|}{2.3. Flujo alternativo} & 2.3.1. Si se retiran las placas de la cuba antes de que el procesos finalice, o el valor de corriente cae debajo de un valor mínimo aceptable por mas de 1 minuto se considera una situación anormal y se emite una alarma.  \\ \cline{ 2- 2}
\multicolumn{ 1}{|l|}{} & 
2.3.2. Si no se restablece el parámetro por mas de 30 segundos los contadores se vuelven a cero y se deja una alarma encendida, sino retorna al proceso al momento en que fue interrumpido. \\ \hline
3. Requerimiento especial & El sistema debe estar configurado en el Modo 2.\\ \hline
4. Pre condiciones  & Sistema energizado e inicializado.  \\ \hline
5. Pos condiciones & Queda en modo espera controlando los parámetros según los requerimientos RFTEM112,121,131,153? \\ \hline
\end{tabular}
\end{flushleft}
\caption{Descripción caso de uso en Modo 2}
\label{caso_uso_func_2}
\end{table}



% Appendix A

\chapter{Esquemáticos} % Main appendix title
\label{AppendixB} % For referencing this appendix elsewhere, use \ref{AppendixA}

\begin{sidewaysfigure}[!ht]
	\centering
	\includegraphics[width=0.9\textwidth]{Figures/Appendices/sch_mainPoncho}
	\caption{Esquemático jerárquico del poncho EduCiaa.}
	\label{fig:schMainPoncho}
\end{sidewaysfigure}


\begin{sidewaysfigure}[h!]
	\centering
	\includegraphics[width=1\textwidth]{Figures/Appendices/sch_inpTempTerm}
	\caption{Esquemático entradas de termocupla y termistor.}
	\label{fig:schEntradas}
\end{sidewaysfigure}


\begin{sidewaysfigure}[h!]
	\centering
	\includegraphics[width=1\textwidth]{Figures/Appendices/sch_outAnalogDigital}
	\caption{Esquemático de salidas digitales con relays.}
	\label{fig:schSalidas}
\end{sidewaysfigure}

\begin{sidewaysfigure}[h!]
	\centering
	\includegraphics[width=1\textwidth]{Figures/Appendices/sch_conect_ciaa}
	\caption{Esquemático de conectores de expansión con la eduCIAA.}
	\label{fig:schSalidas}
\end{sidewaysfigure}



%% Appendix Template

\chapter{ Requerimientos Funcionales y no Funcionales } % Main appendix title

\label{AppendixC} % Change X to a consecutive letter; for referencing this appendix elsewhere, use \ref{AppendixX}

\begin{enumerate}

\item Requerimientos Funcionales
\begin{enumerate}

\item Temperatura (RFTEM)
\begin{enumerate}
\item El sistema medirá la temperatura con un resolución de XX, cada YY segundos.
\item El sistema mantendrá la temperatura controlada según los parámetros configurados por el usuario.
\item El sistema elevará la temperatura a través de la activación de una salida digital conectada a una resistencia.
\item La activación de la resistencia se implementará a través de una ventana Smith Trigger para evitar la intermitencia y generación de ruido en las líneas de alimentación principal. Cruce por 0.
\item En caso de que la temperatura se exceda de rango de considerar interrumpir el proceso y emitir una alarma.
\item El sistema almacenará al menos XX valores de temperatura en un archivo en memoria flash.
\end{enumerate}

\item Energía (RFENE)
\begin{enumerate}
\item El software medirá la corriente total (DC) entregada al proceso de electrólisis cada XX segundos, con YY de resolución.
\item El software medirá la tensión aplicada (DC) entre los bornes del electrólisis cada XX segundos con YY de resolución.
\item El sistema almacenará al menos XX valores de corriente y tensión en un archivo en memoria flash.
\item Los rangos de valores óptimos de tensión y corriente serán tomados de los parámetros de lote ingresados por el usuario y mostrados por pantalla para su configuración manual en la fuente de alimentación principal.
\end{enumerate}

\item Conductividad (RFCOND)
\begin{enumerate}
\item El software calculará a través de la corriente y tensión medidas en el tanque de galvanización, la conductividad de la solución salina cada XX segundos con YY de resolución.
\item El software deberá compensar las eventuales desviaciones de la conductividad óptima a través de la activación de XX válvulas de aditivos.
\end{enumerate}

\item Interfaces Hombre-Máquina (RFHMI)
\begin{enumerate}
\item El sistema contará con una pantalla estado del sistema con variables a definir.
\item El sistema contará con un método de ingreso de parámetros de lote a procesar.
\item Deberá permitir ingresar parámetros en modo manual y otros en modo codificado.
\item Deberá brindar a través de una interfaz Ethernet los históricos almacenados en memoria flash de variables del proceso que necesiten ser auditadas tras una etapa o tras el proceso completo. El máximo de registros será de XX número de puntos en formato YY.
\end{enumerate}

\begin{enumerate}
\item Tiempos (RFTI)
\item El software llevará un conteo del tiempo entre cada baño en las bateas, desde el momento que se inicia hasta el final del proceso.
\item En cada etapa deberá avisar y esperar a que un operario habilite la iniciación de la siguiente etapa.
\end{enumerate}

\item Niveles de bateas (RFNB)	
\begin{enumerate}
\item Evaluará que los niveles dentro del galvanizador estén dentro de los rangos permitidos de operación.
\item En caso de algún nivel crítico se emitirán alarmas y se considera la interrupción del proceso.
\end{enumerate}
\end{enumerate}

\item Requerimientos de Interfaz
\begin{enumerate}

\item Temperatura (RITEM)
\begin{enumerate}
\item La temperatura será medida a través de un sensor analógico de tolerancia XX.
\item La temperatura se medirá en los tanques XX, lo que arroja un total de YY numero de entradas analógicas independientes.(+xxAI)
\end{enumerate}

\item Energía (RIENE)
\begin{enumerate}
\item Tendrá un sensor de alta corriente del tipo XX conectado en una entrada analógica, para la corriente de galvanizador. (+1AI)
\item Tendrá un sensor de tensión de tipo XX conectados a los bornes de galvanizador. (+1AI)
\end{enumerate}

\item Conductividad (RICOND)
\begin{enumerate}
\item Tendrá XX dispositivos dosificador/es conectados a salidas digitales. (+xxDO)
\end{enumerate}

\item Interfaces Hombre-Máquina (RIHMI)
\begin{enumerate}
\item Debe mostrar información a través de un puerto VGA/HDMI con una taza de refresco menor a XX segundos. (+1USB)
\item Tomará de una entrada serie USB los valores de lote. (+1USB)
\item Accionara a través de una salida digital una alarma sonora/lumínica en caso de algún tipo de falla. (+1DO, +1AO) 
\item Contará con uno o dos pulsadores a fin de poder detener y accionar el procesos de galvanización, conectados a una/dos entradas binarias.(+1/2DO)
\item Dispondrá de una conexión remota a través de Ethernet. (+1ETH)
\end{enumerate}

\item Niveles de bateas (RINB)
\begin{enumerate}
\item Tendrá XX sensores de nivel conectados a entradas digitales. (+xxDI)
\end{enumerate}

\end{enumerate}

\item Requerimientos no Funcionales (RNF)
\begin{enumerate}
\item Deberá ser probada la funcionalidad a través de un banco de pruebas que se ajuste al comportamiento del sistema. 
\end{enumerate}

\item Restricciones de Diseño (RD)
\begin{enumerate}
\item De los requerimientos de interfaz se resume que como mínimo el hardware deberá contar con las siguientes interfaces:
Entradas analógicas:(AI) >= 3	\\
Entradas digitales: (DI) >= 3	\\
Salidas analógicas: (AO) >= 1	\\
Salidas digitales:  (DO) >= xx	\\
Puerto USB:	  		(USB)  = 2	\\
Puerto RED:	  		(ETH)  = 1	\\
\end{enumerate}

\item Requerimientos a Futuro (RAF)
\begin{enumerate}
\item Brindar información acerca de si es necesario realizar una limpieza de sistema. Se puede utilizar como parámetro el número de procesos que se ejecutaron 	 	 	
\item Deberá permitir loguearse antes de iniciar el proceso, así tener un responsable de operación.
\item Deberá interactuar con una cinta de transportación automática que llevará las placas de una batea a otra. Accionara los motores (paso a paso?) de transporte y de elevación.
\item Si no se respetan los tiempos el sistema deberá dejar asentado el técnico y las acciones manuales ejecutadas a fin de tener un histórico antes posibles fallas en el lote.
\end{enumerate}


%----------------------------------------------------------------------------------------
%	BIBLIOGRAPHY
%----------------------------------------------------------------------------------------

\Urlmuskip=0mu plus 1mu\relax
\raggedright
\printbibliography[heading=bibintoc]

%----------------------------------------------------------------------------------------

\end{document}  
